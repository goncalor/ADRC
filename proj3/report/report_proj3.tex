\documentclass[10pt,a4paper]{article}
\usepackage[utf8]{inputenc}
\usepackage[portuguese]{babel}
\usepackage[T1]{fontenc}
\usepackage{amsmath}
\usepackage{amsfonts}
\usepackage{amssymb}
\usepackage{graphicx}
\usepackage[portuguese]{algorithm2e}
\usepackage{comment}
\usepackage{hyperref}
\usepackage[all]{hypcap}
\usepackage{listings}
\usepackage{lstautogobble}
%\usepackage[margin=2.2cm]{geometry}

\makeatletter	% let the hacks begin
\newcommand{\nosemic}{\renewcommand{\@endalgocfline}{\relax}}	% Drop semi-colon ;
\makeatother

\author{Gonçalo Ribeiro e Ricardo Amendoeira}
\title{Conectividade em Digrafos}
\begin{document}

\begin{titlepage}

	\begin{center}

		\includegraphics[width=6cm]{./logoIST}\\[2.5cm]

		\textsc{\LARGE Algoritmia e Desempenho\\[5mm]
		em Redes de Computadores}\\[2cm]

		{ \huge \bfseries Conectividade em Digrafos \\[2.5cm] }


		\noindent
		\begin{center} \Large
			Gonçalo Ribeiro, 73294\\[5mm]

			Ricardo Amendoeira, 73373\\[4.5cm]

			\textit{Docente: Prof. João Luís Sobrinho}

		\end{center}

		\vfill

		{\large 9 de Dezembro de 2014}


	\end{center}

\end{titlepage}
\pagebreak

\section{Caminhos Disjuntos entre Dois Nós}
Na primeira parte do projecto é-nos pedido um algoritmo que dado um digrafo, um nó de origem e um de destino determine qual o número mínimo de ligações que é preciso quebrar para que o nó de origem deixe de conseguir chegar ao nó de destino.

Podem existir inúmeros caminhos entre o nó de origem e o de destino mas sabemos que se esses caminhos não forem disjuntos em termos de arestas ao quebrarmos uma aresta comum quebramos todos caminhos. Parece portanto óbvio que este problema é equivalente a encontrar o número de caminhos disjuntos que permitem chegar do nó origem ao nó de destino. Há que notar também que como estamos na presença de um digrafo não é recíproco ir de um nó A para B ou de B para A.

O problema pode ser traduzido para um problema de fluxos máximos em que cada ligação tem capacidade unitária e em que o fluxo permitido é ou unitário ou nulo. Para resolver este problema podemos usar o algoritmo de Ford--Fulkerson ou o de Edmonds--Karp. Optámos por nos basear no segundo algoritmo, que faz as procuras com BFS. O pseudo-código da nossa implementação pode ser visto no Algoritmo~\ref{algo:EdmondsKarp}.

\lstset{emph={graph, src, dest, nr_disjoint, path}, emphstyle=\itshape} 
\begin{algorithm}[h]
\caption{algoritmo para contagem de caminhos disjuntos num digrafo}
\label{algo:EdmondsKarp}
\begin{lstlisting}[linewidth=0.95\linewidth]
count_disjoint(graph, src, dest):
  nr_disjoint = -1
  while there is a path from src to dest:
    increment nr_disjoint
    path = BFS(graph, src, dest)
    if a path was found:
      reverse links in path

  return nr_disjoint
\end{lstlisting}
\end{algorithm}

Sendo $n$ o número de nós e $m$ o número de arestas, o algoritmo de Edmonds--Karp é $O(n m^2)$ e a complexidade do Algoritmo~\ref{algo:EdmondsKarp} é a mesma. Verificámos que é possível melhorar este resultado usando o algoritmo de Goldberg que é $O(n^2 m)$. No entanto considerámos que os grafos a considerar são relativamente esparsos e que portanto a melhoria não seria significativa. Segundo \cite{IntrotoAlgorithms3rd} existe ainda o algoritmo ``relabel-to-front'' que é $O(n^3)$.

\section{Imunidade a Quebras}
Nesta secção é pedido que se calculem para cada k (número natural), a fracção de pares de nós que são separados quebrando apenas k ligações. A primeira parte deste projecto calcula o número de caminhos disjuntos entre um par de nós, o que equivale ao número de ligações que têm de ser quebrados para separar o par (um por cada caminho disjunto). 

Assim, o método utilizado para resolver a 2ª parte foi correr o Algoritmo~\ref{algo:EdmondsKarp} para cada par de nós (em ambos os sentidos, uma vez que o grafo é direccionado), incrementando o número de pares separados com k ligações cada vez que se descobre um par com k caminhos disjuntos entre si, obtendo no fim o numerador da fracção para cada valor de k. O denominador da fracção é obtido somando o valor de pares separados para cada valor de k, o que corresponde ao número de caminhos disjuntos no grafo. O pseudo-código deste processo pode ser visto no Algoritmo~\ref{algo:Redundancy} 

\lstset{emph={graph, src, dest, min_k, k}, emphstyle=\itshape} 
\begin{algorithm}[h]
\caption{algoritmo que determina a k-conectividade de um digrafo}
\label{algo:Redundancy}
\begin{lstlisting}[linewidth=0.95\linewidth, mathescape]
Redundancy(graph):
  int separated_by[k]
  for each src in graph:
    for each dest in graph:
      if src $\neq$ dest:
        k = count_disjoint(graph, src, dest)
        if k != 0:
         separated_by[k] = separated_by[k] + 1 

  return separeted_by
\end{lstlisting}
\end{algorithm}

O Algoritmo~\ref{algo:Redundancy} usa o Algoritmo~\ref{algo:EdmondsKarp} (Edmonds-Karp), de complexidade $O(n m^2)$, $n(n-1)$ vezes, levando a uma complexidade de $O(n^3 m^2)$ para o Algoritmo~\ref{algo:Redundancy}.

\section{k-conectividade de um Grafo}
No último problema deste trabalho é pedido um algoritmo que calcule a k-conectividade de um digrafo em termos de arestas (\textit{k-edge-connectivity)}, ou seja dado um digrafo fortemente conexo qual o número mínimo de ligações que é preciso remover para que o grafo deixe de ser fortemente conexo. Para além disso queremos encontrar um conjunto de ligações que uma vez quebradas fazem o grafo deixar de ser conexo.

Da definição anterior segue que para um digrafo ser k-conexo todos os pares de nós têm que ser pelo menos k-conexos, ou seja é necessário que existam pelo menos $k$ caminhos disjuntos de \emph{cada nó} para cada um dos outros. Esta condição poderia ser relaxada no caso de estarmos na presença de um grafo não direccionado i.e.\ num grafo não direccionado basta que \emph{um nó} tenha $k$ caminhos disjuntos para cada um dos outros. Esta diferença deve-se ao facto de que num grafo não direccionado os caminhos disjuntos de A para B são os mesmos que de B para A, mas num digrafo estes caminhos não são coincidentes e podem ser em número diferente.

Tendo em conta que o objecto de estudo deste trabalho são digrafos, um algoritmo possível para determinar a k-conectividade é veririficar qual o mímino de caminhos disjuntos que existe entre cada par de nós. O par que tiver entre si o menor número de caminhos disjuntos impõe esse número como o valor de $k$. O Algoritmo~\ref{algo:k-conectividade} faz uso deste raciocínio.

\lstset{emph={graph, src, dest, min_k, k}, emphstyle=\itshape} 
\begin{algorithm}[h]
\caption{algoritmo que determina a k-conectividade de um digrafo}
\label{algo:k-conectividade}
\begin{lstlisting}[linewidth=0.95\linewidth, mathescape]
k_connectivity(graph):
  min_k = $\infty$
  for each src in graph:
    for each dest in graph:
      if src $\neq$ dest:
        k = count_disjoint(graph, src, dest)
        if k == 0:
          return not strongly connected
        if k < min_k:
          min_k = k

  return min_k
\end{lstlisting}
\end{algorithm}

O Algoritmo~\ref{algo:k-conectividade} usa o Algoritmo~\ref{algo:EdmondsKarp} $n(n-1)$ vezes, pelo que a complexidade do primeiro será $O(n^3 m^2)$.

Para descobrir um exemplo de $k$ ligações que quebradas façam a rede não ser conexa usámos o raciocínio de que quando o Algoritmo~\ref{algo:EdmondsKarp} termina as ligações que fazem parte de caminhos disjuntos estarão invertidas em comparação com o grafo original. Consequentemente se do conjunto de ligações invertidas escolhermos $k$ que pertençam a caminhos disjuntos obtemos um conjunto de ligações que se quebradas tornam o grafo desconexo. Uma forma simples de escolher ligações de caminhos disjuntos é escolher $k$ ligações invertidas de entre as que saem da origem ou de entre as que entram no destino.

\section{Conclusão}
Este 3º mini-projecto tem grande utilidade para ajudar a identificar as zonas mais vulneráveis de uma rede (no caso de perder certas ligações) e consequentemente a aumentar a sua robustez. Foi notada a falta da existência um algoritmo conhecido para grafos direccionados que resolva o problema da $edge-connectivity$ de forma mais eficiente que usando $n(n-1)$ vezes um algoritmo de $min-cut/max-flow$  de um no para outro como o de $Edmonds--Karp$ usado neste mini-projecto. Sendo ainda uma área de estudo, é possível que uma solução assinptóticamente mais eficiente seja apresentada no futuro.  Como falado nas aulas, um problema de $min-cut$ é muito próximo de um problema de $max-flow$, pelo que este mini-projecto pode ser facilmente alterado para resolver esse tipo de problemas. O visualizador de grafos criado no 2º mini-projecto foi actualizado para o tipo de redes usadas neste trabalho.


%\section{Visualizador de Grafos}

\bibliographystyle{plain}
\nocite{slidesADRC}
\bibliography{IntrotoAlgorithms3rd,slidesADRC}	% no spaces between commas!

\end{document}
